
\chapter{Sécurité informatique et Cybersécurité} 
\minitoc
\thispagestyle{empty}
\newpage
\section{Introduction : }
%\parindent=0.2cm 

Dans un monde de plus en plus connecté où les progrès technologiques avancent à grande vitesse, où les gens, les entreprises, les organismes, les pays et même les objets sont de plus en plus connectés.
Dans un tel monde, il est absolument essentiel d'assurer une cybersécurité efficace. Tant que l'utilisation des TIC progressent, les risques augmenteront aussi. Il s'agit donc de répondre à des défis et des cybermenaces qui ne cessent d'évoluer, ce qui exige que tous les acteurs soient au courant des facteurs de risque, qu'ils disposent des capacités nécessaires et qu'ils prennent les mesures appropriées en matière de prévention et de résolution. Pour garantir un cyberespace sûr, résilient et sécurisé, les pays doivent explorer la réalité complexe et multisectorielle de la sécurité ce qui soulève des questions stratégiques, techniques, juridiques, politique, et qui suppose l'établissement d'une collaboration multisectorielle et internationale \cite{securite}.\\

Aujourd’hui notre mode de vie ultra connecté ne peut se passer des objets connectes et comme une connexion à un réseau rend possible une intrusion. Le fait d’interconnecter des objets augmentent d’autant plus la surface d’attaque aux cyber attaquants nous rendant ainsi beaucoup plus vulnérables. Cette multiplicité d’objets, de connexions, et de règlementation fait qu’aujourd’hui qu’il est difficile d’appliquer un seul modèle de sécurité. Le cyber espace est donc un monde à hauts risques d’attaques \cite{dangercyberespace}. Le mot anglais « Security » signifie une résistance à une malveillance, se traduit en français par « sûreté », alors que le mot « safety » signifie une résistance à une panne \cite{ref4ex}.\\

La sécurité informatique est l’ensemble des moyens techniques, organisationnels, juridiques et humains nécessaires mis en oeuvre pour réduire la vulnérabilité d’un système contre les menaces accidentelles ou intentionnelles. C’est une branche de la technologie de l’information qui étudie et met en oeuvre les menaces et les vulnérabilités des systèmes informatiques.\\
La cybersécurité assure la protection du cyberespace contre les cybers menaces. Le terme cybersécurité n’est pas synonyme de sécurité informatique\cite{ref1} . La cybersécurité s’applique aux systèmes interconnectés, du fait que l’information numérique à protéger voyage à travers eux et y réside en eux tant disque la sécurité informatique s'occupe des systèmes informatisés liés ou non à internet ou à des réseaux\cite{ref1e2}.
\section{Les objectifs de sécurité informatique : }
La notion de sécurité d’un système informatique s’exprime généralement en termes de disponibilité (D), d’intégrité (I) et de confidentialité (C)\cite{ref4}. Ces critères (dits critères DIC) sont les objectifs de sécurité de base dont leurs mises en oeuvre permettent d’atteindre un certain niveau de sécurité. En plus de ces fonctions s’ajoutent d’autres services de sécurité complémentaires pour confirmer la véracité ou l’authenticité d’une action ou d’une ressource (notion d’authentification) ou encore pour prouver l’existence d’une action à des fins de non-répudiation ou d’imputabilité, ou de traçabilité.\\
\cia
\subsection{La confidentialité : }
La confidentialité consiste à rendre l'information inintelligibles aux acteurs illégitimes d’une transaction ou d’une communication. Elle garantit l’anonymat des données en limitant leurs accès par le chiffrement et l’authentification.\\
La confidentialité assure que seules les personnes autorisées peuvent accéder à l’information. Les politiques d’entreprise devront limiter l’accès à l’information au personnel autorisé et garantir que seules ces personnes autorisées consultent ces données. Les données peuvent être compartimentées selon le niveau de sécurité ou de
sensibilité de l’information.\\
Par ailleurs, les employés doivent suivre une formation pour comprendre les bonnes pratiques en matière de protection des informations sensibles pour se protéger et pour protéger l’entreprise contre les attaques. Il existe deux méthodes complémentaires permettant d’assurer la confidentialité :
\begin{itemize}[label=\textbullet]
	\item Limiter et contrôler l’accès aux données afin que seules les personnes habilitées à les lire ou à les modifier puissent le faire.
	\item Rendre inintelligibles les données en les chiffrant de telle sorte que les personnes qui ne sont pas autorisées à les déchiffrer ne puissent les utiliser. Cette tache est réalisée en utilisant le chiffrement (cryptographie).
\end{itemize}
\subsection{L’intégrité :}
Vérifier l’intégrité des données consiste à déterminer si les données n'ont pas été altérées durant la communication. L’intégrité représente l’exactitude, la cohérence et la fiabilité des données pendant tout leur cycle de vie. Le critère d’intégrité des ressources physiques et logiques (équipements, données, traitements, transactions, services) assure qu’elles demeurent intactes, qu’elles n’ont pas été détruites ou modifiées à l’insu de leurs propriétaires tant de manière intentionnelle, qu’accidentelle.\\

Préserver l’intégrité des ressources et s’assurer que des ressources sont intègres sont l’objet de mesures de sécurité. Ainsi, se prémunir contre l’altération des données et avoir la certitude qu’elles n’ont pas été modifiées collabore à la qualité des prises de décision basées sur celles-ci. Si en télécommunication, l’intégrité des données relève essentiellement de problématiques \cite{ref4}liées au transfert de données, elle dépend également des aspects purement informatiques de traitement de l’information (logiciels d’application, systèmes d’exploitation, environnements d’exécution, procédures de sauvegarde, de reprise et de restauration des données). L’intégrité des données est mise en oeuvre par des mécanismes cryptographiques et les signatures numériques pour s’assurer que les données n’ont pas été victimes d’écoute ou d’altération lors de leur transfert par des cyberattaques
\subsection{La Disponibilité :}
La disponibilité d’une ressource est relative à la période de temps pendant laquelle le service qu’elle offre est opérationnel. Elle garantit la continuité à l’accès à un service ou à des ressources. Le volume potentiel de travail susceptible d’être pris en charge durant la période de disponibilité d’un service détermine la capacité d’une ressource à être utilisée. Il ne suffit pas qu’une ressource soit disponible, elle doit pouvoir être utilisable avec des temps de réponse acceptables. Sa disponibilité est indissociable de sa capacité à être accessible par l’ensemble des ayants droit (notion d’accessibilité).\\

La disponibilité des services, systèmes et données est obtenue par une maintenance des équipements, la réparation des matériels, la mise à jour des systèmes d’exploitation et des logiciels, la création de sauvegardes ainsi que l’utilisation d’équipements ou de service de sécurité. Les attaques de types DoS et DDOS  peuvent causer l’indisponibilité d’un service ou d’une ressource. Celles-ci, sont possibles si les procédures de sauvegarde et de restitution ainsi que les supports de mémorisation associés ne sont pas gérées correctement ou s’il y a malveillance. Une politique de sauvegarde et des systèmes de détection d’attaques de types DDOS doivent être mis en oeuvre pour éviter le risque d’indisponibilité d’un service ou d’une ressource.
\subsection{La traçabilité : }
L’enregistrement des activités permettent la traçabilité des événements et leur analyse. Garder la mémoire des actions survenues permet notamment de reconstituer et de comprendre ce qui s’est passé lors d’incidents afin d’améliorer la sécurité, d’éviter que des erreurs ne se répètent ou d’identifier des fautifs. Cela autorise par exemple d’analyser le comportement du système et des utilisateurs à des fins d’optimisation, de gestion des incidents et des performances ou encore d’audit.\\
L’enregistrement des actions et événements permet également d’enrichir les bases de données qui permettent de développer des applications de surveillance, de détection et de réaction aux incidents, en particulier à l’aide des techniques issues de l’intelligence artificielle.
\subsection{L’authentification et l’identification :}
L'identification consiste à attribuer une identité unique à un utilisateur. Elle permet de répondre à la question : Qui êtes-vous ?\cite{ref5}. Pour se faire l'utilisateur utilise un identifiant unique qu’on nomme Compte d'accès (Nom d'utilisateur ou Login en anglais). L’authentification consiste à assurer la véracité de l'identité d'un utilisateur. Elle permet de répondre à la question : êtes-vous réellement cette personne ? \cite{ref5} Pour se faire l’utilisateur utilise un authentifiant ou code secret (mot de passe ou password en anglais) dont lui seul à la connaissance.\\
L’identification et l’authentification des ressources et des utilisateurs permettent d’associer la réalisation d’une action à une entité qui pourra en être tenue responsable et éventuellement en rendre compte. Un contrôle d'accès permet l'accès à des ressources uniquement aux personnes autorisées à travers un mot de passe crypté.
\authenti
\subsection{La non-répudiation et l’imputabilité : }
La non-répudiation de l'information est la garantie qu'aucun des correspondants ne pourra nier une transaction, c'est-à-dire de garantir à chacun des correspondants que son partenaire est bien celui qu’il prête être en d’autres termes assurer la preuve d’origine ou de destination de message.\\
L’imputabilitéest la possibilité d’attribuer la responsabilité d’une infraction à un individu. Elle peut être réalisée par un ensemble de mesures garantissant l’enregistrement fiable d’informations pertinentes relatives à un événement. La non-répudiation et l’imputabilité sont assurer en utilisant les algorithmes de chiffrement asymétriques.
\section{Application de la sécurité informatique : }
La sécurité contient une variété de contextes et dépends du domaine d’application. On distingue 6 catégories de sécurité en fonction du domaine d’application  :
\begin{itemize}[label=\textbullet]
\item \textbf{Sécurité matérielle physique et environnementale};
\item \textbf{Sécurité logique et applicative};
\item \textbf{Sécurité de l’information} ;
\item \textbf{Sécurité de l’exploitation} ;
\item \textbf{Sécurité des réseaux };
\item \textbf{Cybersécurité}.
\end{itemize}
\imageAPS
\subsection{La cybersécurité :}
Le mot Cybernétique vient du grec « kuberneïn » qui signifie diriger ou gouverner, terme repris en 1948 par le mathématicien Norbert Wiener ancien professeur du Massachusetts Institute de Technologie (MIT) qui l’a publié pour la première fois dans son ouvrage intitulé Cybernetics, or Control and Communication in the Animal and the Machine\cite{cybernetics}.\\
Plusieurs décennies plus tard l’auteur de science-fiction William Gibson utilise le terme de cyberespace dans son roman le Neuromanien. Il s’agit d’une trilogie qui a pour personnage central un voleur de données. Celui-ci est à mesure d’établir des connexions entre son esprit et un réseau mondial liant des ordinateurs entre eux \cite{neuromacer}.\\
Au fuir du temps le préfixe \textit{cyber} va participer ainsi à la construction de nouveaux mots lui donnant une nouvelle définition. Cyber est un préfixe servant de créer de mots relatifs à l’utilisation d’internet et à cette société de l’information qui a vu le jour à la fin du XXème. Aujourd’hui il y a plus de 40 mots débutants par Cyber dont : Cybersécurité, cybernétique, cyberespace, cybercafé, cybercrime, cyberattaque, cyberdéfense, cyberguerre, cyberterroriste, cybermonde etc.\\Avec l’utilisation extensif d’Internet des objets générant de nouveaux types de menaces dont seule la cybersécurité peut y faire face.\\
La cybersécurité assure la protection du cyberespace contre les cybers attaques. Elle concerne la sécurité des systèmes accessibles via le cyberespace(internet)et englobe la sécurité de l’information, la sécurité des réseaux et des environnements connectés. Le cyberespace est un espace virtuel relatif à notre espace naturel en d’autres termes un ensemble d’infrastructures numériques, de données et services misent en réseaux mais contrairement à la terre, à la mer, à l’air et à l’espace-extra atmosphérique, le cyberespace est une pure création de l’être humain qui ne relève pas de la nature \cite{ref4}. Le terme cybersécurité est large et englobe chaque élément, de la sécurité de l’ordinateur à la reprise de l’activité après sinistre et la formation des utilisateurs \cite{ref8}
\subsection{La sécurité des réseau : }
La sécurité d’un réseau consiste à protéger un réseau informatique contre les intrus, qu'il s'agisse d'attaques ciblés ou de logiciels malveillants en d’autres termes protéger les équipements, les applications et les données du réseau contre les cybers attaques. Elle utilise plusieurs services et protocoles de défense allant de la couche physique à la couche application. Pour se faire plusieurs approches sont utilisés dans le modèle TCP/IP comme décrit ci-dessous :
\begin{itemize}[label=\textbullet]
\item Couche transport (protocoles TLS/SSL, SSH),
\item Couche internet (protocole IP Sec);
\end{itemize}
\subsubsection{Protocoles TLS/SSL :}
\textbf{TLS} de l’acronyme \textbf{T}ransport \textbf{L}ayer \textbf{S}ecurity et \textbf{SSL} de l’acronyme \textbf{S}ecure \textbf{S}ocket \textbf{L}ayer, sont des protocoles de chiffrement qui garantissent la sécurité des échanges de données via un réseau informatique. Ils sont largement utilisés pour la sécurisation des communications sur internet.\\
Ils utilisent les techniques de chiffrements asymétriques et des algorithmes de cryptage comme RSA, ECC, pour assurer la confidentialité, l’intégrité des données ainsi que l’authentification et l’identification des utilisateurs.
\subsubsection{Le protocole IPSec :}
IP Sec pour Internet Protocol Security est une suite de protocoles normalisés par L’Internet Engineering Task Force (IETF) qui fournit des services de sécurisations des données dans un réseau IP au niveau de la couche réseau. Fondé dans le but d’assurer la sécurité du protocole IPv6 et a été réadapter par la suite au protocole IPv4. Il assure les critères de confidentialité, d’authentification et d’intégrité des données échangées à travers un réseau IP ainsi qu’à la création de réseau privé virtuel (VPN).\\
L’IPSec peut fonctionner en deux : mode transport et mode tunnel. Le mode transport est utilisé pour les communications de bout en bout (Host to Host). Le mode tunnel est utilisé pour les configurations passerelle à passerelle ou passerelle à hôte (Gate-to-Gate ou host-to-Gate)\cite{ref9}.\\

En plus de ces protocoles il existe plusieurs moyens classiques pour protéger un réseau informatique dont : les firewalls, la segmentation du réseau, les anti-virus et anti-malwares, les contrôles d’accès réseaux, les zones démilitarisés (DMZ) dont nous détaillerons dans la section mécanismes de sécurité.
\subsection{La sécurité logique et applicative :}
La sécurité des applications se concentre sur la protection des logiciels contre les menaces. Une application compromise pourrait fournir un accès aux données qu'elle est destinée à protéger. Une sécurité réussie commence au stade de la conception, bien avant le déploiement d'un programme ou d'un périphérique.\\
La sécurité logique fait référence à la réalisation de mécanismes de sécurité par logiciel contribuant au bon fonctionnement des programmes, des services offerts et à la protection des données. Elle s’appuie généralement sur :
\begin{itemize}
\item La qualité des développements et l’implémentation des logiciels et des tests de sécurité ;
\item Une mise en oeuvre adéquate de la cryptographie pour assurer intégrité et confidentialité ;
\item Des procédures de contrôle d’accès logique, d’authentification ;
\item Des procédures de détection de logiciels malveillants, de détection d’intrusions et d’incidents ;
\item La sécurité applicative comprend le développement pertinent de solutions logicielles (ingénierie du logiciel, qualité du logiciel) ainsi que leur intégration et exécution harmonieuses dans des environnements opérationnels.
\end{itemize}
\subsection{La sécurité des informations : }
La sécurité de l'information est un ensemble de stratégies de gestion et politiques de sécurités visant à protéger, détecter, recenser et contrer les menaces ciblant les données.  Elle vise à protéger des données et s'applique à tous les aspects de la sûreté, l'intégrité, la disponibilité et la confidentialité, la garantie, et la protection d'une donnée ou d'une information quelle que soit sa forme, tant en stockage qu'en transit \cite{ref8}. Une donnée est la représentation d’une information.

Le système d’information étant le maillon de l’entreprise, l’information qui y transite doit être impérativement protéger contre le vol, la destruction et la falsification d’information pouvant causer d'énorme pertes à l’entreprise. Une classification des données permet de qualifier leur degré de sensibilité (normale, confidentielle, etc.) et de les protéger en fonction de ce dernier
\subsection{La sécurité d'exploitation : }
La sécurité de l'exploitation est un ensemble de stratégies de gestion et de politiques de sécurités visant à assurer le bon fonctionnement opérationnel des systèmes informatiques. Cela comprend la mise en place d'outils et de procédures relatifs aux méthodologies d'exploitation, de maintenance, de test, de diagnostic, de gestion des performances, de gestion des changements et des mises à jour\cite{ref4}. Elle s'appuie sur les différents mécanismes et services de sécurités suivants :
\begin{itemize}[label=\textbullet]
\item Une gestion des systèmes d’exploitation, des configurations et des mises à jour ;
\item Gestion des incidents et suivi jusqu’à leur résolution ;
\item Une politique de sauvegarde, de secours, de continuité et de tests;
\item Gestion des contrats de maintenance ;
\item Une analyse des fichiers de journalisation et de comptabilité 
\end{itemize}
\section{Anatomie d'une attaque :}
Une attaque informatique est l'exploitation d'une faille de sécurité d'un système dans le but de le violer et de l'utiliser pour réaliser des actions malveillantes. Une attaque peut se faire par un individu ou un groupe d'individus, contre l'ordinateur d'un individu ou d'un groupe de personnes morale ou physique. Elle se déroule en générale en cinq actions fréquemment appelés les 5 P  \cite{ref5p} : Probe, Penetrate, Persist, Propagate, Paralyze.
\begin{itemize}
\item\textbf{Probe} (phase d'analyse): consiste à l'identification de la cible. Pour se faire l'attaquant se sert des outils comme whois, Arin, DNS lookup pour collecter le maximun d'information sur la cible en vue d'identifier d'éventuelles vulnérabilités.
\item\textbf{Penetrate} (phase de pénétration) : consiste à l'utilisation des informations et vulnérabilités récoltées pour pénétrer un réseau.
\item\textbf{Persist} (persister) : une fois le réseau infiltré, le pirate procédera à la création de droit administrateur en vue d y revenir facilement. Pour cela, il installera par exemple des back doors ou cheval de Troie en français.
\item\textbf{Propagate} (propager) : le réseau  infiltré, l'accès à un compte administrateur, Le pirate pourra ensuite  explorer le réseau afin de trouver de nouvelles cibles sur le réseau local. 
\item\textbf{Paralyze} (paralyser) : cette phase est la plus dangereuse. Elle est l'étape finale qui consiste à l'exécution  de l'attaque. pour cela le pirate peut utiliser le serveur ou la machine infiltrée pour attaquer une autre machine, détruire des données ou encore endommager le système d’exploitation dans le but de le nuire
\end{itemize}
\section{Les types de menaces et d’attaques : }
Une menace est une cause potentielle d’incident, qui peut provoquer un dommage sur un système et ayant un impact sur ses fonctionnalités, son intégrité ou sa disponibilité. Une cybermenace est une menace qui s’exprime via le cyberespace, qui peut toucher tout système connecté à Internet dont sa concrétisation par une cyberattaque, peut affecter le bon fonctionnement des ordinateurs, des réseaux de télécommunication et de tous les services et activités humaines qui en dépendent\cite{cybermenace}. Les cybermenaces sont le plus souvent associées à l’usage malveillant des technologies Internet et à la cybercriminalité. De nombreuses cyberattaques existent, elles recouvrent des réalités diverses en fonction des cibles touchées, de leurs impacts, finalités, origines et auteurs.
\subsection{La cybercriminalité : }
Selon Colin ROSE \cite{ref10} La cybercriminalité est la troisième grande menace pour les grandes puissances, après les armes chimiques, bactériologiques, et nucléaires. Le cout global de la cybercriminalité est estimé à 6000 milliards de dollars et triplera le nombre d’emplois non pourvu en cybersécurité d’ici 2021 \cite{ref1112}.\\
\textbf{La cybercriminalité} est toute infraction impliquant l’utilisation des technologies informatiques. Elle comprend des acteurs uniques ou des groupes ciblant des systèmes à des fins de gain financier ou de perturbation. Malgré que les entreprises et les forces de l’ordre tentent de s’attaquer au problème croissant, le nombre de cybercriminels continue de croître, profitant de l’anonymat d’Internet.\\
La cybercriminalité se divise en trois grandes catégories \cite{ref1112} : la cybercriminalité individuelle, la cybercriminalité contre la propriété et la cybercriminalité gouvernementale 
\begin{description}
\item[La cybercriminalité individuelle] est une catégorie de cybercriminalité dont le criminel utilise des informations et programmes malveillants pour arriver dans le but de nuire à un individu.
\item[La cybercriminalité contre la propriété] est la catégorie de cybercriminalité dont le criminel s’usurpe (vol d’identité) de l’identité d’une personne. Une fois le criminel possède les informations personnelles volées il procède à des achats, des élévations de privilèges afin d’accéder à des informations confidentielles ou au chantage.
\item[La cybercriminalité gouvernementale] est l’infraction la plus grave qui consiste à un crime contre le gouvernement, par exemple le piratage de sites web gouvernementaux ou de la diffusion de propagande. Ces genres de cyberattaques sont menés par des terroristes ou dans le cadre d’une cyberguerre entre nations.
\end{description}
\subsection{DDoS (Distributed Denied of Service) : }
Le DDoS est une attaque informatique ayant pour but de rendre indisponible un service, d'empêcher les utilisateurs légitimes d'un service de l'utiliser. La ressource peut être une seule machine (comme un serveur), un groupe de machines(comme un pool de serveurs dédiés), voire un réseau. Le DDoS est actuellement l’une des attaques la plus dangereuse pour les états, les entreprises causant beaucoup de perte d’argent et même en vie humaine si elle est bien structurée et visant des infrastructures critiques\cite{refados}. Voilà pourquoi nous avons fait le choix de nous focaliser sur ce type d’attaque. Le chapitre suivant est consacré entièrement à l’attaque DDoS.
\subsection{L'ingénierie sociale : }
L’ingénierie sociale est un ensemble des techniques de manipulation psychologique ou d’exploitation comportementale d’un individu, ou d’un groupe d’individus, par des personnes malfaisantes dont le but est l’incitation à amoindrir, contourner ou supprimer les mesures de sécurité d’un système par ce ou ces individus \cite{ref13}.\\
L’ingénierie sociale se sert en général des comportements ou de la personnalité particulière de leurs cibles tels que la naïveté, les émotions personnelles, les centres d’intérêt, L’adresses e-mail, la serviabilité, la confiance, le respect, la fierté, la reconnaissance dans le but de s’emparer de leurs accès informatiques. Elle procède comme suite : 
\begin{itemize}
	\item Collecte d’information,
	\item Etablissement de relations,
	\item Exploitations des vulnérabilités identifiées,
	\item Et ensuite l’exécution.
\end{itemize}
Les types d’attaques d’ingénierie sociale les plus utilisées sont : l’hameçonnage, le vishing, etc.
\subsubsection{L’Hameçonnage : }
L’hameçonnage ou phishing en anglais est l’une les plus courantes des attaques d’ingénierie sociale. C’est un courriel frauduleux ou un faux site conçu en usurpant l’identité d’un individu ou d’une organisation pour inciter leurs cibles à révéler des informations privées (nom d’utilisateur, mots de passe, information de cartes de crédit, etc.) ou télécharger des logiciels malveillants.\\

Les courriels d'hameçonnage reposent sur des tactiques de peur, comme des courriels urgents de votre banque ou d'une autre institution financière, ou de votre patron, comme des offres de produits bon marché ou difficiles à trouver ou votre sens du devoir envers votre patron. Ça pourrait être aussi un faux site web contrôler par l’attaquant, usurpant d’un site officiel incitants les utilisateurs à fournir leurs informations personnelles.
\subsubsection{Le vishing : }
Le vishing ou attaque par téléphone est la version téléphonique de l’hameçonnage et la plus facile à mettre en oeuvre, c’est une technique frauduleuse utiliser par les pirates pour récupérer des informations (généralement bancaires) auprès d’utilisateurs de téléphone portable.\\

Les hackers qui utilisent cette technique préparent soigneusement leur personnage et leur discours, au préalable et ensuite appeler leur cible dans le but d’obtenir des renseignements le plus rapidement possible. Certains pour parfaire leur crédibilité, utilisent un magnétophone ou une cassette préalablement enregistrée de bruits de bureau, ou encore utiliser un matériel qui change le timbre de la voix pour imiter celle d'une secrétaire ou d’un patron.

\subsection{Attaque par brute force et par dictionnaire : }
L’objectif de ces attaques est généralement le même; deviner le bon mot de passe ou de décrypter un texte en utilisant des tentatives, des techniques et des logiciels de décryptage. Elle se base sur la philosophie \textit{n’importe quel mot de passe est crackable ce n’est qu’une question de temps.}\\

\textbf{L’attaque par force brute} consiste à tester de façon exhaustive toutes les combinaisons possibles de caractères (alphanumérique, symbole) de manière à trouver un mot de passe valide. Elle se base sur le fait que n’importe quel mot de passe est crackable ce n’est qu’une question de temps.\\

\textbf{L’attaque par dictionnaire} consiste à cracker un mot de passe en se basant sur document répertoriant des mots courants, des prénoms, des noms d’animaux ou des objets etc.
Les outils d'attaque par force brute peuvent demander des heures, voire même des jours ou des années, de calcul même avec des machines équipées de processeurs puissants. Ainsi, une alternative consiste à effectuer une « attaque par dictionnaire » souvent vue comme un complément de l’attaque par brute fore. En effet, la plupart du temps les utilisateurs choisissent des mots de passe ayant une signification réelle. Avec ce type d'attaques, un tel mot de passe peut-être craquer en quelques minutes
\subsection{Attaque par Homme du milieu :}
Attaque par homme du milieu (HDM) ou man-in-the-middle en anglais (MITM) est une attaque par laquelle un attaquant accède aux communications entre deux noeuds légitimes, sans qu’aucune de ces deux noeuds ne s’en rende compte. L’attaquant peut lire le contenu de la communication, parfois les modifier.
\imgMiM
Pour se faire l’attaquant passe généralement par une connexion internet et doit donc être capable de recevoir les messages (les décrypter s’ils sont chiffrés) des deux parties et d'envoyer des réponses à une partie en se faisant passer pour l'autre \cite{ref14}.

\subsection{Logiciels Malveillants (Malware) :}
Un logiciel malveillant est un type de logiciel conçu pour obtenir un accès non autorisé ou pour endommager un ordinateur. Il comporte un ensemble de programmes conçus par un pirate pour être implantés dans un ordinateur à l’insu de l’utilisateur. I1 regroupe les virus, vers, spywares, keyloggers, chevaux de Troie, backdoors etc.
\subsubsection{Les virus : }
Un virus informatique est un programme ou un code malveillant qui est chargé dans votre ordinateur à votre insu sans votre autorisation. Certains virus sont seulement désagréables, mais la plupart sont destructeurs et sont conçus pour infecter les systèmes vulnérables et en prendre le contrôle.\\

Les virus sont généralement attachés à un fichier exécutable ou un document Word. Ils se propagent souvent via le partage de fichiers en P2P, de sites Internet infectés et le téléchargement
de pièces jointes. A l’interaction avec ces fichiers infectés par exemple le clic sur un fichier, l’ouverture d’un fichier ou à l’exécution d’un programme, le virus s’exécute automatiquement et s’installe dans l’ordinateur de la victime. Une fois qu’un virus pénétré dans votre système, il restera en dormance jusqu’à l’activation du programme ou du fichier hôte infecté, qui à son tour activera le virus et lui permettra de s’exécuter et de se reproduire sur votre système \cite{ref13}.
\subsubsection{Chevaux de Troie : }
Le cheval de Troie est un logiciel qui ne se reproduit pas et permet d’ouvrir une « porte » sur l’ordinateur de la victime pour en prendre ultérieurement le contrôle ou activer à distance des programmes nocifs appelés « malwares ».\\ 

Encore appelé « Trojan Horse » en anglais, ce type de logiciel n’est rien d’autre que le véhicule, celui qui fait entrer le programme malveillant à l’intérieur de la machine. Il n’est pas nuisible en lui-même car il n’exécute aucune action, si ce n’est de permettre l’installation du vrai programme malveillant.
Très présent dans les pièces jointes de messagerie, ces programmes sont souvent destinés au vol de données personnelles, et notamment financières.
\subsubsection{Les keyloggers : }
Il s’agit d’enregistreurs de frappes de touche du clavier d’un ordinateur dont les informations sont ensuite adressées au pirate agissant à distance. Il lui est ainsi possible de connaître les informations sensibles de sa victime comme des références bancaires ou toutes autres données qui lui permettraient de commettre ultérieurement une escroquerie.\\

Certains keyloggers sont capables d’enregistrer les URL visitées, les courriers électroniques consultés ou envoyés, les fichiers ouverts, voire de créer une vidéo retraçant toute l’activité de l’ordinateur infecté. Ils peuvent prendre la forme soit, d’un logiciel informatique soit, d’un support matériel. Dans le premier cas, il s’agit d’un processus furtif écrivant les informations captées dans un fichier caché. Dans le second cas, il s’agit alors d’un dispositif intercalé entre la prise clavier de l’ordinateur et le clavier.
\subsubsection{Les vers : }
Les vers tout comme les virus sont les deux exemples de logiciels malveillants les plus répandus et les plus connus. Ce sont des programmes malveillants capables de s’autorépliquer sur les ordinateurs ou via les réseaux informatiques et d’infecter les ordinateurs à l’insu de leurs utilisateurs. La plupart peuvent causer d’importants préjudices.\\

Dans la mesure où chaque copie du ver informatique peut à son tour s’autorépliquer, les infections peuvent se propager très rapidement. Il existe plusieurs catégories et sous-catégories de virus et vers informatiques. On peut citer les vers d’e-mail ou les vers de messagerie instantanée, les virus envoyés sous forme de pièces jointes ou via les réseaux de partages de fichier P2P.\\

La différence principale entre un virus et un vers est du fait que les virus nécessitent un programme hôte actif ou un système d’exploitation actif et déjà infecté pour s’exécuter, alors que les vers sont autonomes capables de s’autoreproduire et de se propager via les réseaux informatiques sans intervention humaine \cite{ref13}.
\subsubsection{Les logiciels espions : }
Encore appelés « Spyware » en anglais, ils correspondent à un terme générique désignant les logiciels espions qui s’introduisent dans un système informatique afin de recueillir à des fins commerciales le profil d’un utilisateur au regard de sa navigation sur le réseau Internet, voire le cas échéant obtenir des informations personnelles comme les références d’une carte bancaire, d’un permis de conduire ou tout autre document personnel et sensible.\\

Ces logiciels espions sont souvent inclus dans des logiciels gratuits et s’installent généralement à l’insu de l’utilisateur en même temps qu’il télécharge le logiciel en question. Ils sont souvent développés par des sociétés proposant de la publicité sur Internet. Pour permettre l’envoi ultérieur de publicité ciblée, il est nécessaire de bien connaître sa cible. Cette connaissance est grandement facilitée par ces logiciels espions \cite{ref13}.
\subsubsection{Les botnets : }
Contraction de robot et réseau, un « botnet » un terme générique qui désigne un groupe d’ordinateurs, de quelques milliers à plusieurs millions, contrôlés par un pirate à distance. Ce sont en réalité des programmes informatiques destinés à communiquer avec d’autres programmes similaires pour l’exécution de différentes tâches.\\
La plus connue consiste à prendre le contrôle à distance, en exploitant une faille de sécurité via un cheval de Troie ; par exemple, des milliers d’ordinateurs zombies forment un réseau de milliers de robots appelés communément des “botnets“.\\
Ces milliers d’ordinateurs contrôlés seront autant de relais pour permettre des attaques puissantes puisque les pirates pourront alors depuis leur domicile diffuser des codes malveillants tout en cachant leur identité. Les enquêteurs, dans le meilleur des cas, arriveront sur des ordinateurs dont les propriétaires sont également des victimes. Ces botnets représentent aujourd’hui une réelle menace pour notre société \cite{ref13}.
\subsubsection{Ransomwares : }
Un ransomware ou Rançongiciel en français est un type de logiciel malveillant, prenant en otage les données d'un individu ou d'une entreprise. Il est conçu pour extorquer de l'argent en bloquant l'accès aux fichiers ou au système informatique jusqu'à ce que la rançon soit payée. Les rançongiciels sont apparus pour la fois en Russie en 2005 \cite{ref13} et se sont répandus dans le monde entier principalement aux Etats-Unis, en Australie ou en Allemagne. Généralement le Ransomware s’infiltre à travers un fichier téléchargé ou reçu par email et chiffre les données et fichiers de la victime. Une fois une machine infectée il est capable d’infecter les autres machines dans le même réseau d’où la nécessité de débrancher les machines infectées de rançongiciels du réseau de l’entreprise.
\ransom
Le paiement de la rançon ne garantit pas forcement que les fichiers seront récupérés. On distingue deux principales formes de ransomware :
\begin{description}
 \item[Le ransomware Locker : ] ce type de ransomware verrouille l’accès et vous empêche d’utiliser les fonctionnalités de base de votre ordinateur. En général vous serez encore en mesure d’interagir avec la demande de rançon afin de procéder au paiement, mais votre ordinateur vous sera inutile pour toutes les autres fonctionnalités. Ce type de ransomware est moins dangereux car il n’affecte par les fichiers critiques de votre ordinateur.
 \item[Le ransomware crypto : ]  ce type de ransomware chiffre vos données critiques (documents, images, vidéos), provoquant la panique chez leurs victimes. Ils n’affectent pas les fonctionnalités de base de l’ordinateur mais vos fichiers critiques, vos fichiers seront visibles et cryptés vous limitant ainsi l’accès. Les attaquants utilisant ce type de ransomware ont tendance à lancer un compte à rebours à leur demande de rançon menaçant leurs victimes de supprimer les fichiers après une certaine date ce qui le rend plus dangereux.\\
Les exemples courants de ransomwares sont : locky, Wannacry, Bad Rabbit, Rhuk etc…
 \end{description} 
\section{Les services et mécanismes de sécurité : } 
Un service de sécurité est un service qui augmente la sécurité des traitements et des échanges de données d’un système. Un service de sécurité utilise un ou plusieurs mécanismes de sécurité \cite{ref15}. Un mécanisme de sécurité est un mécanisme qui est conçu pour détecter, prévenir et lutter contre une attaque informatique.
Quelques bonnes pratiques pour améliorer la sécurité des systèmes informatiques :
\subsection{Audit de sécurité :}
L’audit de sécurité est l’identification des points de vulnérabilité d’un système. Il ne détecte pas les attaques ayant déjà eu lieu, ou lorsqu'elles auront lieu. L’audit de sécurité est généralement assuré par un expert de la sécurité informatique au sein de l’entreprise. Il conçoit les mécanismes de sécurité et attaques le système informatique de l’entreprise pour tester sa robustesse.
\subsection{Journalisation (logs) :}
La journalisation est l’enregistrement des activités de chaque acteurs (les utilisateurs). Il permet de constater que des attaques ont eu lieu, de les analyser et ainsi que de faire en sorte qu'elles ne se reproduisent pas.
\subsection{Antivirus :}
Les antivirus sont des logiciels dont le rôle est de protéger l’ordinateur contre les logiciels (ou fichiers potentiellement exécutables dangereux) néfastes. Il ne protège pas le réseau contre un intrus qui emploie un logiciel légitime, ou contre un utilisateur légitime qui accède à une ressource alors qu'il n'est pas autorisé à le faire.
\subsection{Utilisation de VPN :}
Un VPN (Virtual Private Network) est un tunnel sécurisé permettant la communication entre deux entités y compris au travers de réseaux peu sûrs comme peut l’être le réseau Internet. Un VPN permet de créer une liaison virtuelle entre deux noeuds physiques distants de manière transparente pour les utilisateurs concernés. Les données envoyées au travers de ces liaisons virtuelles sont chiffrées, ceci garantit aux utilisateurs d’un VPN qu’en cas d’interception malveillante les données soient illisibles.
\subsection{Systèmes de détection d'intrusion (IDS) :}
Un systèmes de détection d’intrusion (IDS) comme son nom l'indique c'est un système mis en œuvre pour détecter les intrusion dans un ordinateur ou dans un réseau.
Il existe depuis 1980 \cite{refids} et fait référence à tous les processus utilisés pour découvrir les utilisations non autorisées du réseau ou des objets du réseau . Ceci est réalisé grâce à un logiciel spécialement conçu dans le seul but de détecter une activité inhabituelle ou anormale.\\ nous présenterons en détails les systèmes de détections d'intrusion dans le chapitre quatre. 

\subsection{Les pare feu : }
Un pare-feu (logiciel ou matériel) du réseau dont le rôle est de contrôler les communications qui le traversent. Il a pour fonction de faire respecter la politique de sécurité du réseau, celle-ci définissant quels sont les communications autorisés ou interdits. Il n'empêche pas un attaquant d'utiliser une connexion autorisée pour attaquer le système et ne protège pas le réseau contre une attaque venant du réseau intérieur (qui ne le traverse pas).
\subsection{La reprise après sinistre et la continuité des activités : }
La reprise après sinistre et la continuité des activités définissent la façon dont une organisation réagit à un incident de cybersécurité ou à tout autre événement entraînant la perte d'opérations ou de données. Les politiques de reprise après sinistre dictent la manière dont l'organisation restaure ses opérations et ses informations pour retrouver la même capacité opérationnelle qu'avant l'événement. La continuité des activités est le plan sur lequel l'organisation s'appuie tout en essayant de fonctionner sans certaines ressources.
\subsection{La formation des utilisateurs finaux : }
Aborde la formation du personnel de l’entreprise sur les notions de sécurité de l'information vu que le facteur humain constitue la plus grande vulnérabilité en terme de sécurité informatique. N'importe qui peut accidentellement introduire un virus dans un système par ailleurs sécurisé en ne respectant pas les bonnes pratiques de sécurité. En général les entreprises mettent en places des politiques de sécurités qui contiennent ces bonnes pratiques de sécurité :
\begin{itemize}
\item Apprendre aux utilisateurs à supprimer les pièces jointes suspectes,
\item Ne pas brancher les clés USB non identifiées,
\item Utilisation de mot de passe fort respectant les normes de sécurités,
\item Une bonne maintenance du système informatique,
\item Désignation d’un responsable de sécurité,
\item La mise en place d’une politique de sauvegarde (back up) adaptée et redondante etc.
\end{itemize}
\section{Conclusion : }
Dans ce chapitre nous avons défini et décrit les différents aspects de la sécurité informatique et de la cybersécurité. La sécurité informatique est un domaine très vaste et devient de plus en plus complexe à cause de la nature constante d’évolution des risques de sécurité et la complexité des systèmes informatique.
Avec un nombre accru de points d'entrée pour les attaques, d'avantage de stratégies de sécurisation des actifs numériques sont nécessaires pour protéger les systèmes d'informations. L'un des éléments les plus problématiques de la la sécurité informatique est la veille technologique, Suivre les changements et les progrès continus des attaques et mettre à jour les pratiques pour les protéger contre les potentielles vulnérabilités .\\

Dans le prochain chapitre, nous exposerons les attaques DDoS, en passant en revue son fonctionnement, les différentes catégories d’attaques DDoS, les motivations derrières ses attaques DDoS ainsi que quelques mesures de prévention.
 







