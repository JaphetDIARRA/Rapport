\selectlanguage{french}
\begin{abstract}
\thispagestyle{plain}
\setcounter{page}{3}
%\setcounter
%\parindent=0.2cm	
\begin{singlespace}
L’inf{\kern0pt}luence grandissante des réseaux informatiques et le besoin d’interconnecter des objets a conduit à une nouvelle forme d’internet qu’est le réseau des Objets connectés(IoT). Dans ce réseau les objets connectés s’échangent des informations pour répondre à un but bien déf{\kern0pt}ini. Cette collaboration  des objets connectés ouvre de nouvelles portes d’at{\kern0pt}taques aux hackers qui ef{\kern0pt}fectuent des at{\kern0pt}taques de plus en plus sophistiquées.\\

Parmi ces nouvelles portes d'at{\kern0pt}taques le Déni de Service Distribué(DDoS est considéré comme la plus grande menace visant l'IoT, à cause de son niveau de sécurité faible et aux nombreuses vulnérabilités des objets connectés. Le réseau IoT comporte des informations sensibles et souvent des systèmes critiques d’où la nécessité d’assurer la confidentialité, l'intégrité et la disponibilité des données échangées. Cela implique de pouvoir prédire le comportement des objets et des données à f{\kern0pt}in d’éviter l’introduction de programmes malveillants par des cybers at{\kern0pt}taquants.\\

A ce problème s’ajoute les limitations des espaces de stockages de certains objets connectés rendant la gestion de leur sécurité complexe ainsi que le problème de la disponibilité des objets connectés facilement enfreint par des at{\kern0pt}taques de types DDoS vue leur simplicité de mise en oeuvre.\\

La problématique de Cybersécurité en ce qui concerne les appareils connectés est un facteur primordial que doit prendre en charge les dif{\kern0pt}férents acteurs et entreprises avant d'adopter ce nouveau concept de réseau IoT.\\


Ce projet de mémoire propose une approche résiliente pour l'identif{\kern0pt}ication et la détection des at{\kern0pt}taques DDoS dans les réseaux IoT en vue de minimiser leurs intrusions, en utilisant les techniques d’apprentissage automatique basé sur le Deep Learning. Ainsi, cette approche pour détecter les attaques DDoS emploient séquentiellement l'auto-encoder(AE) et le réseaux de neurones profond(DNN). Ce projet utilise le langage de programmation Java avec le framework deeplearning4j. Nous avons évalué notre modèle avec deux jeu de données bien connus, à savoir IoT Botnet et NSL-KDD. Cette approche s'avère donner un taux de réussite (accuracy) plus élevé et un taux de faux positifs plus bas.\\


\textbf{Mots clés : Internet des Objets(IoT), Système de Detection d'Intrusion (IDS), cybersécurité, Deni de Service Distribué(DDoS)}

\end{singlespace}

\end{abstract}
%%%%%%%%%%%%%%%%%%%%%%%%%%%%%%%%%%%%%%%
%abstract 
\selectlanguage{english}
\begin{abstract}
\thispagestyle{plain}
\setcounter{page}{4}
\parindent=0.2cm
\begin{singlespace}
The growing inf{\kern0pt}luence of computer networks and the need to interconnect objects has led to a new form of the internet which is the Internet of Things(IoT) network. In this network, connected Things exchange information to meet a well-defined goal. This collaboration of connected Things or devices opens new doors of at{\kern0pt}tack for hackers who carry out increasingly sophisticated attacks. \\

Among these new at{\kern0pt}tacks doors, the Distributed Denial of Service(DDoS) is considered to be the greatest threat targeting the IoT, due to the low level of security and the numerous vulnerabilities of connected objects. The IoT network supports critical systems, hence the need to ensure reliability in their operations in order to avoid possible loss of data, material or even loss of human life. This implies being able to predict the behavior of objects and data exchanged in order to avoid the introduction of malicious programs by cyber at{\kern0pt}tackers. \\ 

In addition to this problem, there are the limitations of the storage spaces of certain connected objects making management their  security complex as well as the problem of the availability of connected objects easily violated by DDoS at{\kern0pt}tacks given their simplicity of implementation. \\

The issue of Cybersecurity regard to Connected devices is an essential factor that must be taken into account by the various actors and companies before adopting this new concept of IoT network given its recent popularity.\\

This thesis project proposes a resilient approach for the identif{\kern0pt}ication and detection of DDoS attacks in IoT networks in order to minimize their intrusions, using machine learning techniques based on Deep Learning. Thus, this approach for detecting DDoS attacks sequentially employs Auto-Encode (AE) and Deep Neural Network (DNN). This project uses the Java programming language with the deeplearning4j framework. We evaluated our model with two well-known datasets, namely IoT Botnet and NSL-KDD. This approach is found to give a higher accuracy rate and a lower false positive rate. \\


\textbf {Keywords: Internet of Things (IoT), Intrusion Detection System (IDS), cybersecurity, Distributed Denial of Service (DDoS)}
\end{singlespace}

\end{abstract}